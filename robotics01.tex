\documentclass{beamer}
\usepackage{graphicx, multirow, array, hyperref, listings}
\usepackage[utf8]{inputenc}
\usepackage{etoolbox}

% There are many different themes available for Beamer. A comprehensive
% list with examples is given here:
% http://deic.uab.es/~iblanes/beamer_gallery/index_by_theme.html
% You can uncomment the themes below if you would like to use a different
% one:

\usetheme{Boadilla}
\usecolortheme{whale}
% \makeatletter
% \patchcmd{\beamer@sectionintoc}
%   {\vfill}
%   {\vskip 2ex}
%   {}
%   {}
% \makeatother

\setbeamertemplate{sections/subsections in toc}[sections numbered]

\hypersetup{
  colorlinks=true,
  linkcolor=blue,
  urlcolor=cyan
}
%\parskip=1.5ex

\title{Biomimetics \& Dexterous Manipulation Lab}

% A subtitle is optional and this may be deleted
\subtitle{Stanford University}

\author[X. S\'{a}nchez]{Xavier~S\'{a}nchez~D\'{i}az \\ 1170065}
% - Give the names in the same order as the appear in the paper.
% - Use the \inst{?} command only if the authors have different
%   affiliation.

\institute[ITESM] % (optional, but mostly needed)
{
Robotics
}
% - Use the \inst command only if there are several affiliations.
% - Keep it simple, no one is interested in your street address.

\date{January 17, 2017}
% - Either use conference name or its abbreviation.
% - Not really informative to the audience, more for people (including
%   yourself) who are reading the slides online

\subject{Robotics}
% This is only inserted into the PDF information catalog. Can be left
% out. 

% If you have a file called "university-logo-filename.xxx", where xxx
% is a graphic format that can be processed by latex or pdflatex,
% resp., then you can add a logo as follows:

% \pgfdeclareimage[height=0.5cm]{university-logo}{university-logo-filename}
% \logo{\pgfuseimage{university-logo}}

% Delete this, if you do not want the table of contents to pop up at
% the beginning of each subsection:
% \AtBeginSubsection[]
% {
%   \begin{frame}<beamer>{Outline}
%     \tableofcontents[currentsection,currentsubsection]
%   \end{frame}
% }


% Let's get started
\begin{document}

\begin{frame}
\titlepage
\end{frame}



\begin{frame}[allowframebreaks=1]{Outline}
\tableofcontents[hideallsubsections=true]
  % You might wish to add the option [pausesections]
\end{frame}



% Section and subsections will appear in the presentation overview
% and table of contents.
\section{Current Projects}

\subsection{Adhesion and Applications}

\begin{frame}{Adhesion and Applications}{BDML: Current Projects}

\begin{itemize}
  \item Crawling and climbing robots
  \item Adhesive and manufacturing methods
  \item UAV perching
  \item Space junk grasping
  \item Manipulation in space
\end{itemize}
\end{frame}

\begin{frame}{Mobile Manipulation}{BDML: Current Projects}
\begin{itemize}
  \item Underwater hand and grasp analysis
  \item Tactile sensing for hands and feet
\end{itemize}
\vspace{3ex}
\begin{block}{Multi-limbed climbing}
Multi-limbed climbing also involves grasping surfaces and manipulating the robot's own body.
\end{block}
\end{frame}

\begin{frame}{Medical Robots compatible tools}{BDML: Current Projects}
\begin{block}{Medical Robots}
Needles embedded with optical fibers for realtime measurement of bending deflections and tip forces.
\end{block}
\end{frame}

\begin{frame}{Multi-modal Robots}{BDML: Current Projects}
\begin{block}{Multi-modal Robots}
Bio-inspired robots that can transition from jumping to flying to climbing vertical walls.
\end{block}

\begin{itemize}
  \item Fixed wing perching robots
  \item \alert{Quadrotor perching robots}
  \item JumpGliding
\end{itemize}
\end{frame}

\begin{frame}{Wearable Haptics}{BDML: Current Projects}
\begin{block}{Wearable Haptics}
Dynamic gait analysis through wearable sensors and feedback devices to reduce the chance of injury or delay the progression of osteoarthritis.
\end{block}

\vspace{3ex}

Some automotive haptics are also being researched.
\end{frame}

\begin{frame}{Tunable compliance and damping}{BDML: Current Projects}

  \begin{block}{Tunable compliance and damping}
  Electroactive polymer actuators with electrically-tunable stiffness and damping for dynamic systems.
  \end{block}

  Some research about Manufacturing and prototyping methods is also being conducted since it's an important part of all projects.
\end{frame}

\section{SCAMP}

\begin{frame}{SCAMP}{Stanford Climbinbg and Aerial Manuevering Platform}
\begin{block}{SCAMP}
The Stanford Climbinbg and Aerial Manuevering Platform is the first robot that's able to fly, passively perch, climb, and take off. It operates outdoors on rough surfaces like concrete and stucco, using only onboard sensing and computation.
\end{block}

\begin{figure}[h]
  \centering
  \includegraphics[width=0.5\textwidth]{scamp}
  \label{fig:scamp}
\end{figure}
\end{frame}

\section{Quadrotor Perching Robots}

\begin{frame}{Quadrotor Perching Robots}{BDML: Multi-modal robots}
\begin{block}{Grants}
Army Research Lab (MCE 13-4.4) \& National Science Foundation (IIS-1161679).
\end{block}

\begin{block}{Collaboration}
Stanford, University of Pennsylvania, University of Maryland \& MIT
\end{block}

\begin{block}{Faculty}
Mark Cutkosky, Vijay Kumar, Sean Humbert, Russ Tedrake.
\end{block}
\end{frame}

\begin{frame}{Quadrotor Perching Robots}{BDML: Multi-modal robots}
\begin{block}{More at}
\texttt{bdml.stanford.edu/Main/DynamicRotorcraftPerchingMechanisms}
\end{block}

\begin{figure}[h]
  \centering
  \includegraphics[width=0.5\textwidth]{rotorcraft}
  \label{fig:rotorcraft}
\end{figure}
\end{frame}

\section{Publications}

\begin{frame}{Publications}{BDML: Multi-modal robots}

\begin{itemize}
  \item Pope, Morgan, "Creatures of two worlds : small robots and hybrid aerial-terrestrial locomotion," \textbf{PhD thesis}, July 2016.
  \item Jiang, H., Pope, M.T., Estrada, M.A., Edwards, B., Cuson, M., Hawkes, E.W., Cutkosky, M.R., "Perching Failure Detection and Recovery with Onboard Sensing," IEEE/RSJ IROS 2015.
  \item Wissa, A., Han, A. and Cutkosky, M.R., "Wings of a Feather Stick Together," 'Biomimetic and Biohybrid Systems, Lecture Notes in Artificial Intelligence, Wilson, S.P., Verschure, P..F.M.J., Mura, A and Prescott, T. (eds.), Springer, Vol. 9222 2015, pp. 123-134.
  \item Estrada, M.A., Hawkes, E.W., Christensen, D.L., and Cutkosky, M.R., "Perching and Crawling: Design of a Multimodal Robot," IEEE/ICRA 2014. \textbf{(Best Paper Finalist)}
\end{itemize}

\end{frame}

\begin{frame}{Publications}{BDML: Multi-modal robots}

\begin{itemize}
  \item Jiang, H., Pope, M.T., Hawkes, E.W., Christensen, D.L., Estrada, M.A., Parlier, A., Tran, R., and Cutkosky, M.R., "Modeling the Dynamics of Perching with Opposed-Grip Mechanisms," accepted for IEEE/ICRA 2014.
  \item Alexis Lussier Desbiens, Alan T. Asbeck and Mark R. Cutkosky, "Landing, Perching and Taking Off from Vertical Surfaces", International Journal of Robotics Research, March 2011 vol. 30 no. 3, pp 355-370.
  \item Alexis Lussier Desbiens, Alan Asbeck and Mark R. Cutkosky, "Hybrid Aerial and Scansorial Robotics", ICRA, May 2010, Anchorage, Alaska.
\end{itemize}


\begin{block}{More Publications}
More publications at \texttt{bdml.stanford.edu/Main/Publications}
\end{block}

\end{frame}

\end{document}