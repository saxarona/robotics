\documentclass[titlepage, letterpaper, fleqn]{article}
\usepackage[utf8]{inputenc}
\usepackage{fancyhdr} % fancy headers, of course!
\usepackage{amsmath} % what do you think?
\usepackage{amsthm} % theorems!
\usepackage{extramarks} % more cute things
\usepackage{enumitem} % i'm not sure...
\usepackage{multicol} % multicolumn...?
\usepackage{amssymb} % more symbols
\usepackage{MnSymbol} % moar symbols?
\usepackage{booktabs} % cool looking tables
\usepackage{tikz} %venn and shizzle
\usepackage{mathrsfs} %math script for calligraphic scripting, I GUESS
\usepackage{listings}
\usepackage{mathtools}
\usepackage[spanish, mexico]{babel}

\topmargin=-0.45in
\evensidemargin=0in
\oddsidemargin=0in
\textwidth=6.5in
\textheight=9.0in
\headsep=0.25in


%
% You should change this things~
%

\newcommand{\mahteacher}{Dr. Ernesto Rodríguez Leal}
\newcommand{\mahclass}{Robótica}
\newcommand{\mahtitle}{Homework 02}
\newcommand{\mahdate}{\today}

\newcommand{\spacepls}{\vspace{5mm}}
\renewcommand\qedsymbol{\(\blacksquare\)}
\renewcommand{\ttdefault}{pcr} %so we can get both bold and tt fonts

\newcommand{\bigO}{\mathcal{O}} %you should be inside a math environment
\let\bs\mathbf

%
% Header markings
%

\pagestyle{fancy}
\lhead{}
\chead{}
\rhead{}
\lfoot{}
\rfoot{}


\renewcommand\headrulewidth{0.4pt}
\renewcommand\footrulewidth{0.4pt}

\setlength\parindent{0pt}
\setlength\parskip{0.5pt}


%
% Create Problem Sections (stolen directly from jdavis/latex-homework-template @ github!)
%

\newcommand{\enterProblemHeader}[1]{
\nobreak\extramarks{}{Problem \arabic{#1} continued on next page\ldots}\nobreak{}
\nobreak\extramarks{Problem \arabic{#1} (continued)}{Problem \arabic{#1} continued on next page\ldots}\nobreak{}
}

\newcommand{\exitProblemHeader}[1]{
\nobreak\extramarks{Problem \arabic{#1} (continued)}{Problem \arabic{#1} continued on next page\ldots}\nobreak{}
\stepcounter{#1}
\nobreak\extramarks{Problem \arabic{#1}}{}\nobreak{}
}

\setcounter{secnumdepth}{0}
\newcounter{partCounter}
\newcounter{homeworkProblemCounter}
\setcounter{homeworkProblemCounter}{1}
\nobreak\extramarks{Exercise \arabic{homeworkProblemCounter}}{}\nobreak{}

%Solution Environment
% \newenvironment{solution}
% {\renewcommand\qedsymbol{$\square$}\begin{proof}[Solution]}
% {\end{proof}}

% Alias for the Solution section header
\newcommand{\solution}{\textbf{\Large Solution}}

%Alias for the new step section
\newcommand{\steppy}[1]{\textbf{\large #1}}

%
% Homework Problem Environment
%
% This environment takes an optional argument. When given, it will adjust the
% problem counter. This is useful for when the problems given for your
% assignment aren't sequential. See the last 3 problems of this template for an
% example.
%
\newenvironment{homeworkProblem}[1][-1]{
\ifnum#1>0
\setcounter{homeworkProblemCounter}{#1}
\fi
\section{Problem \arabic{homeworkProblemCounter}}
\setcounter{partCounter}{1}
\enterProblemHeader{homeworkProblemCounter}
}{
\exitProblemHeader{homeworkProblemCounter}
}

%
% My actual info
%

\title{
\vspace{1in}
\textbf{Tecnológico de Monterrey} \\
\vspace{0.5in}
\textmd{\mahclass} \\
\large{\textit{\mahteacher}} \\
\vspace{0.5in}
\textsc{\mahtitle}\\
\author{Alicia del Río \and Cristina \and Guillermo Sotelo \and Sergio Sedas \and Xavier Sánchez}
\date{\mahdate}
}

\begin{document}

\begin{titlepage}
\maketitle
\end{titlepage}

%
% Actual document starts here~
%

\section{Cinemática Directa} % (fold)
\label{sec:forward}

Esta sección describe las ecuaciones del análisis de cinemática directa asumiendo que tenemos el diagrama justo como en el paper de Meccanica.
La sección está dividida en subsecciones dependiendo del tipo de ecuaciones que se trate.

\subsection{Lazo cerrado} % (fold)
\label{sec:loop-closure}

% subsection loop-closure (end)

\begin{equation}
    \label{eq:loop-closure}
    p+b_i = a_i + d_i
\end{equation}

\begin{equation}
    \label{eq:p-vector}
    \bs{p} = [p_x,p_y,p_z]^T
\end{equation}

\begin{equation}
    \label{eq:a-vector}
    \bs{a}_i = \bs{R}(Z,\psi_i) \cdot [a,0,0]^T
\end{equation}

\begin{equation}
    \label{eq:b-vector}
    \bs{b}_i = \prescript{G}{}{\bs{R}}_H \cdot \bs{R}(Z,\chi_i) \cdot [b_i, 0, 0]^T
\end{equation}

\begin{equation}
    \label{eq:d-vector}
    \bs{d}_i = \bs{R}(Z,\psi_i) \cdot \bs{R}(\prescript{I}{}{Z},\phi) \cdot \bs{R}(\prescript{II}{}{X}, \theta_{i1})[0,d,0]^T
\end{equation}

\subsection{Matrices de rotación} % (fold)
\label{sec:rotations}

\begin{equation}
    \label{eq:rot_Z_psi}
    \bs{R}(Z,\psi_i) =
    \begin{bmatrix}
        \cos\psi_i & -\sin\psi_i & 0 \\
        \sin\psi_i & \cos\psi_i & 0 \\
        0 & 0 & 1
    \end{bmatrix}
\end{equation}

\begin{equation}
    \label{eq:rot_Z_phi}
    \bs{R}(Z,\phi) =
    \begin{bmatrix}
        \cos\phi_i & -\sin\phi_i & 0 \\
        \sin\phi_i & \cos\phi_i & 0 \\
        0 & 0 & 1
    \end{bmatrix}
\end{equation}

\begin{equation}
    \label{eq:rot_X_theta}
    \bs{R}(X,\theta_{i1}) =
    \begin{bmatrix}
        1 & 0 & 0 \\
        0 & \cos\theta_{i1} & -\sin\theta_{i1} \\
        0 & \sin\theta_{i1} & \cos\theta_{i1}
    \end{bmatrix}
\end{equation}

\begin{equation}
    \label{eq:rot_H_global}
    \prescript{G}{}{\bs{R}}_H =
    \bs{R}(Z,\gamma) =
    \begin{bmatrix}
    u_x & v_x & w_x \\
    u_y & v_y & w_y \\
    u_z & v_z & w_z
    \end{bmatrix}
    =
    \begin{bmatrix}
    \cos\gamma & -\sin\gamma & 0 \\
    \sin\gamma & \cos\gamma & 0 \\
    0 & 0 & 1
    \end{bmatrix}
\end{equation}

% subsection rotations (end)

% section forward (end)
\end{document}