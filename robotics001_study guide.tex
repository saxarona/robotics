\documentclass[titlepage, letterpaper, fleqn]{article}
\usepackage[utf8]{inputenc}
\usepackage{fancyhdr} % fancy headers, of course!
\usepackage{amsmath} % what do you think?
\usepackage{amsthm} % theorems!
\usepackage{extramarks} % more cute things
\usepackage{enumitem} % i'm not sure...
\usepackage{multicol} % multicolumn...?
\usepackage{amssymb} % more symbols
\usepackage{MnSymbol} % moar symbols?
\usepackage{booktabs} % cool looking tables
\usepackage{tikz} %venn and shizzle
\usepackage{mathrsfs} %math script for calligraphic scripting, I GUESS
\usepackage{listings}
\usepackage{mathtools}
\usepackage{caption}
\usepackage[cm]{sfmath}
\usepackage{tcolorbox}


\topmargin=-0.45in
\evensidemargin=0in
\oddsidemargin=0in
\textwidth=6.5in
\textheight=9.0in
\headsep=0.25in


%
% You should change this things~
%

\newcommand{\mahteacher}{Dr. Ernesto Rodríguez Leal}
\newcommand{\mahclass}{Robótica}
\newcommand{\mahtitle}{Cinemática Directa}
\newcommand{\mahdate}{\today}

\newcommand{\spacepls}{\vspace{5mm}}
\renewcommand\qedsymbol{\(\blacksquare\)}
\renewcommand{\ttdefault}{pcr} %so we can get both bold and tt fonts
\renewcommand{\familydefault}{\sfdefault} %The sans-serif font and the like

\newcommand{\bigO}{\mathcal{O}} %you should be inside a math environment
\let\bs\mathbf

%
% Command to add format to bmatrix, directly stolen from
% http://texblog.net/latex-archive/maths/amsmath-matrix/ !!
%

\makeatletter
\renewcommand*\env@matrix[1][*\c@MaxMatrixCols c]{%
  \hskip -\arraycolsep
  \let\@ifnextchar\new@ifnextchar
  \array{#1}}
\makeatother

%
% Header markings
%

\pagestyle{fancy}
\lhead{}
\chead{}
\rhead{}
\lfoot{}
\rfoot{}


\renewcommand\headrulewidth{0.4pt}
\renewcommand\footrulewidth{0.4pt}

\setlength\parindent{0pt}
\setlength\parskip{0.5pt}


%
% Create Problem Sections (stolen directly from jdavis/latex-homework-template @ github!)
%

\newcommand{\enterProblemHeader}[1]{
\nobreak\extramarks{}{Problem \arabic{#1} continued on next page\ldots}\nobreak{}
\nobreak\extramarks{Problem \arabic{#1} (continued)}{Problem \arabic{#1} continued on next page\ldots}\nobreak{}
}

\newcommand{\exitProblemHeader}[1]{
\nobreak\extramarks{Problem \arabic{#1} (continued)}{Problem \arabic{#1} continued on next page\ldots}\nobreak{}
\stepcounter{#1}
\nobreak\extramarks{Problem \arabic{#1}}{}\nobreak{}
}

\setcounter{secnumdepth}{0}
\newcounter{partCounter}
\newcounter{homeworkProblemCounter}
\setcounter{homeworkProblemCounter}{1}
\nobreak\extramarks{Exercise \arabic{homeworkProblemCounter}}{}\nobreak{}

%Solution Environment
% \newenvironment{solution}
% {\renewcommand\qedsymbol{$\square$}\begin{proof}[Solution]}
% {\end{proof}}

% Alias for the Solution section header
\newcommand{\solution}{\textbf{\Large Solution}}

%Alias for the new step section
\newcommand{\steppy}[1]{\textbf{\large #1}}

%
% Homework Problem Environment
%
% This environment takes an optional argument. When given, it will adjust the
% problem counter. This is useful for when the problems given for your
% assignment aren't sequential. See the last 3 problems of this template for an
% example.
%
\newenvironment{homeworkProblem}[1][-1]{
\ifnum#1>0
\setcounter{homeworkProblemCounter}{#1}
\fi
\section{Problem \arabic{homeworkProblemCounter}}
\setcounter{partCounter}{1}
\enterProblemHeader{homeworkProblemCounter}
}{
\exitProblemHeader{homeworkProblemCounter}
}

%
% My actual info
%

\title{
\vspace{1in}
\textbf{Tecnológico de Monterrey} \\
\vspace{0.5in}
\textmd{\mahclass} \\
\large{\textit{\mahteacher}} \\
\vspace{0.5in}
\textsc{\mahtitle}\\
\author{Xavier Sánchez \\
01170065\\
\texttt{xavier.sanchezdz@gmail.com}}
\date{\mahdate}
}

\begin{document}

\begin{titlepage}
\maketitle
\end{titlepage}

%
% Actual document starts here~
%

\section{Kinematics} % (fold)
\label{sec:kinematics}

Kinematics is the study of the movement, in this case, of a robot.

A serial robot, like a robotic arm, has a number of \textbf{links}.
These links form each part of the robot, those which are rigid.
Links are usually referred to using $\bs{n}$.

Between two links, there's always a \textbf{joint}.
Joints can \textbf{rotate} or \textbf{move}.
Joint are usually referred to  using $\bs{j}$.
Revolute joints rotate, while Prismatic joints move.

Any movement o rotation in one dimension is called a \textbf{degree of freedom}, or simply DOF.
It's common to think of a fully functional robot to have at least 6-DOF,
3 for rotations and 3 for movement; one on each dimension.

To calculate the \textbf{dof of a system}, one would apply the following equation:

\begin{equation}
    \label{eq:dof_system}
    DOF = An - Cn
\end{equation}

Where $A$ is the number of parameters of each link $n$,
and $C$ is the number of constraints.
It is, usually, $6n-Cn$, since 6 DOF are expected on each link.
The number of constraints is usually $n-1$

A robot is said to be \textbf{redundant} if it has more links than the DOF of the end-effector.
This is calculated as follows:

\begin{equation}
    \label{eq:dof_redundancy}
    n-m_0
\end{equation}

Where $n$ is the number of links, and $m_0$ are the DOF of the end-effector.

\subsection{Rotation matrices} % (fold)
\label{sec:rotation_matrices}

The \textit{basic} rotations matrices are always the same.
They are multiplied by the desired angle, and give you the output rotation.
The following are the elemental rotation:

\begin{equation}
    \label{eq:rot_x}
    \bs{R}(X,\theta) =
    \begin{bmatrix}
        1 & 0 & 0 \\
        0 & \cos \theta & - \sin \theta \\
        0 & \sin \theta & \cos \theta
    \end{bmatrix}
\end{equation}

\begin{equation}
    \label{eq:rot_y}
    \bs{R}(Y,\theta) =
    \begin{bmatrix}
        \cos \theta & 0 & \sin \theta \\
        0 & 1 & 0 \\
        - \sin \theta & 0 & \cos \theta
    \end{bmatrix}
\end{equation}

\begin{equation}
    \label{eq:rot_z}
    \bs{R}(Z,\theta) =
    \begin{bmatrix}
        \cos \theta & - \sin \theta & 0 \\
        \sin \theta & \cos \theta & 0 \\
        0 & 0 & 1
    \end{bmatrix}
\end{equation}

To get the rotation matrices of a reference frame according to another,
then you need to do the \textbf{dot product} of each axis on both frames,
so you end up with a new matrix that has an element-wise multiplication.

Another interesting feature is that rotation matrices are \textbf{orthonormal}, since their inverse is equal to their transverse, i.e.:

\begin{equation}
    \label{eq:ortho}
    \bs{R}^{-1} = \bs{R}^T
\end{equation}

% subsection rotation_matrices (end)

\subsection{Position} % (fold)
\label{sec:position}

The \textbf{position} of a point, let's say $P$,
with respect to the origin (usually denoted as $O$)
is described by a vector from $O$ to $P$:
simply put, $p$.

To \textbf{map a vector} from one reference system to another,
you need to multiply vector by the rotation matrix from one system to the other, that is:

\begin{equation}
    \label{eq:vector_mapping}
    P_A = \bs{R}P_B
\end{equation}

Where $P_B$ is the vector from point $P$ to reference system $B$,
and $P_A$ is the vector from Point $P$ to reference system $A$.

% subsection position (end)

\subsection{Translation} % (fold)
\label{subsec:translation}

The final position in a \textbf{translation} is obtained when you sum all vectors included in said translation.

\spacepls

\begin{tcolorbox}
REMEMBER KIDDO: if you're gonna operate with vectors (sum, for example) then \textbf{ALL} vectors must be with respect of the \textbf{SAME} system, that is, the origin. \textbf{ALWAYS}.
\end{tcolorbox}

% subsection translation (end)

\subsection{Homogeneous Transform} % (fold)
\label{subsec:homogeneous_transform}

An \textbf{homogeneous transform} is a special matrix which contains the position of the end effector.
This position is expressed as the product of all needed rotations,
and then by its displacements (vectors).
An additional row of 1s or 0s is added: 1 if it's a vector, 0 if it's a rotational matrix.
The homogeneous transform is usually denoted by $\bs{T}$ or $\bs{H}$.

\begin{equation}
    \label{eq:transform_general}
    P_2 = \begin{bmatrix}[ccc|c]
    & \bs{R}(K,\theta) & & Q \\ \hline
    0 & 0 & 0 & 1
    \end{bmatrix} P_1
\end{equation}

This means that a point $P_2$ can be obtained from the product of $P_1$ by its transform matrix:

\begin{equation}
    \label{eq:transforming}
    P_2 = T P_1
\end{equation}

% subsection homogeneous_transform (end)

% section kinematics (end)

\section{Denavit-Hartenberg} % (fold)
\label{sec:denavit}

Denavit-Hartenberg is a representation convention that can be applied to \textbf{serial} robots.
It uses four parameters:

\begin{itemize}
    \item $\bs{\theta}$: a rotation about the $\bs{Z}$-axis.
    \item $\textbf{d}$: the distance over the $\bs{Z}$-axis.
    \item $\textbf{a}$: the length of each common normal (joint offset).
    \item $\bs{\alpha}$: the angle between two successive $\bs{Z}$-axes (joint twist).
\end{itemize}

And it follows a special procedure.
First, start off with these steps:

\begin{enumerate}
    \item Assign joint number $n$ to the first joint.
    \item Assign a local reference frame for each joint before or after the $n$ joint.
\end{enumerate}

Now, continue using these easy steps:

\begin{enumerate}
    \item \textbf{Rotate} about the $Z_n$-axis by $\theta_{n+1}$.
    \item \textbf{Translate} along the $Z_n$-axis by $d_{n+1}$ to make $X_n$ and $X_{n+1}$ colinear.
    \item \textbf{Translate} along the $X_n$ axis a distance of $a_{n+1}$ to bring the origins of $X_{n+1}$ together.
    \item \textbf{Rotate} the $Z_n$-axis about $X_{n+1}$-axis an angle of $\alpha_{n+1}$ to align $Z_n$-axis with $Z_{n+1}$-axis.
\end{enumerate}

\begin{tcolorbox}
REMEMBER KIDDO: the $\bs{Y}$-axis is never used in DH representation!
\end{tcolorbox}

\subsection{D-H General Matrix} % (fold)
\label{sec:d_h_general_matrix}

The general Denavit-Hartenberg matrix is like a magic formula,
you just input your values $a, \alpha, d,$ and $\theta$ and you're done for that transformation.

This is the general matrix of DH parameters:

\begin{equation}
    \label{eq:general_DH}
    T_{n-1 \to n} =
    \begin{bmatrix}[ccc|c]
    \cos \theta_n & -\sin \theta_n & 0 & a_{n-1} \\
    \sin \theta_n \cos\alpha_{n-1} & \cos \theta_n \cos \alpha_{n-1} & - \sin \alpha_{n-1} & -d_n \sin \alpha_{n-1} \\
    \sin \theta_n \sin \alpha_{n-1} & \cos \theta_n \sin\alpha_{n-1} & \cos \alpha_{n-1} & d_n \cos \alpha_{n-1} \\ \hline
    0 & 0 & 0 & 1
    \end{bmatrix}
\end{equation}

This means that the only parameter that needs to be obtained for $n$ is $\theta_n$,
but the rest are parameters from joint $n-1$.

% subsection d_h_general_matrix (end)

% section denavit (end)

\end{document}